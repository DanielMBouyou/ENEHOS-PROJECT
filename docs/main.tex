\documentclass[11pt,a4paper]{article}

% =========================
% ENEHOS — Test & HAHIL Doc
% =========================

% --------- Encodage / Langue ----------
\usepackage[T1]{fontenc}
\usepackage[utf8]{inputenc}
\usepackage[french]{babel}

% --------- Mise en page ----------
\usepackage[a4paper,margin=2.2cm]{geometry}
\usepackage{microtype}
\usepackage{parskip} % évite les gros pavés + spacing propre

% --------- Liens / PDF ----------
\usepackage[hidelinks]{hyperref}
\usepackage{bookmark}

% --------- Images ----------
\usepackage{graphicx}
\usepackage{float}
\usepackage{subcaption} % photos côte à côte
\usepackage{caption}

% --------- Tableaux ----------
\usepackage{array}
\usepackage{tabularx}
\usepackage{booktabs}

% --------- Couleur (minimal) ----------
\usepackage{xcolor}

% --------- Schémas ----------
\usepackage{tikz}
\usetikzlibrary{arrows.meta,positioning,fit,calc,shapes.geometric,shapes.misc}

% (Optionnel si tu veux des schémas de circuits)
\usepackage[european]{circuitikz}

% --------- Blocs visuels (peu de texte, plus de structure) ----------
\usepackage[most]{tcolorbox}
\tcbset{
  colback=white,
  colframe=black!35,
  boxrule=0.6pt,
  arc=2mm,
  left=2mm,right=2mm,top=1mm,bottom=1mm
}
\newtcolorbox{boxnote}{title=\textbf{Note},colframe=black!35}
\newtcolorbox{boxtest}{title=\textbf{Test},colframe=black!35}
\newtcolorbox{boxcheck}{title=\textbf{Checklist},colframe=black!35}

% --------- Listes compactes ----------
\usepackage{enumitem}
\setlist[itemize]{nosep,leftmargin=*,topsep=2pt}
\setlist[enumerate]{nosep,leftmargin=*,topsep=2pt}

% --------- Commandes utiles ----------
\newcommand{\code}[1]{\texttt{#1}}
\newcommand{\imgpath}{images} % dossier d'images Overleaf
\newcommand{\docversion}{v0.3}

% --- Couleurs légères ---
\definecolor{ENEHOSblue}{HTML}{1F4E79}
\definecolor{ENEHOSgray}{HTML}{F5F7FA}
\definecolor{ENEHOSline}{HTML}{B8C2CC}
\definecolor{ENEHOScodebg}{HTML}{0B1020}
\definecolor{ENEHOScodefg}{HTML}{E6EDF3}

% --- Style TikZ (boîtes + flèches uniformes) ---
\tikzset{
  enebox/.style={
    draw=ENEHOSline,
    fill=ENEHOSgray,
    rounded corners=3mm,
    minimum width=52mm,
    minimum height=14mm,
    align=center,
    very thick
  },
  enearrow/.style={-Latex, line width=0.9pt, draw=ENEHOSblue},
  enearrow2/.style={Latex-Latex, line width=0.9pt, draw=ENEHOSblue}
}

% --- Code lisible (fond clair) ---
\usepackage{listings}

\definecolor{codebg}{HTML}{F6F8FA}
\definecolor{codeframe}{HTML}{D0D7DE}
\definecolor{codetext}{HTML}{0B0F14}
\definecolor{codekw}{HTML}{0B61A4}
\definecolor{codecom}{HTML}{2F6F3E}
\definecolor{codestr}{HTML}{8A2BE2}

\lstdefinestyle{enehos}{
  backgroundcolor=\color{codebg},
  basicstyle=\ttfamily\small\color{codetext},
  keywordstyle=\bfseries\color{codekw},
  commentstyle=\itshape\color{codecom},
  stringstyle=\color{codestr},
  numbers=left,
  numberstyle=\tiny\color{black!55},
  numbersep=8pt,
  frame=single,
  rulecolor=\color{codeframe},
  frameround=ffff,
  breaklines=true,
  showstringspaces=false,
  tabsize=2
}

% --------- Titre ----------
\title{\textbf{Projet OZEN}\\\large Plan de tests matériels et HAHIL}
\author{Henry D}
\date{2026-02-16 --- \docversion}

\begin{document}
\maketitle
\tableofcontents
\newpage

% =========================================================
\section{But du document}
\begin{boxnote}
Ce document décrit le \textbf{setup minimal} et la \textbf{procédure de test} :
\begin{itemize}
  \item \textbf{par composant} (XIAO, GNSS L86, BNO085 IMU+magneto+fusion, antenne),
  \item \textbf{par liaison} (UART, I\textsuperscript{2}C, BLE),
  \item en mode \textbf{HAHIL} : Human--AI--Hardware-in-the-Loop.
\end{itemize}
\end{boxnote}

% =========================================================
\section{Vue d'ensemble du système ENEHOS (architecture cible)}
\subsection{Architecture fonctionnelle cible}

\begin{boxnote}
L'architecture cible ENEHOS est la suivante :
\begin{itemize}
  \item \textbf{1 boîtier électronique unique} à la nuque : XIAO nRF52840 + GNSS L86 + BNO085 (IMU 9 axes + magnétomètre + fusion),
  \item \textbf{1 boîtier antenne} au sommet du support,
  \item \textbf{1 boîtier piles AAA},
  \item \textbf{2 boîtiers vibreurs} (gauche / droite).
\end{itemize}
Le boîtier électronique est placé à la nuque pour l'éloigner des moteurs et limiter les perturbations magnétiques.
\end{boxnote}

\begin{figure}[H]
\centering
\begin{tikzpicture}[
  x=1cm,y=1cm,
  scale=0.90,
  transform shape,
  font=\footnotesize,
  outer/.style={draw, rounded corners=2mm, align=center, minimum width=3.4cm, minimum height=1.1cm},
  inner/.style={draw, rounded corners=1.5mm, align=center, minimum width=2.9cm, minimum height=0.9cm, fill=black!3},
  flow/.style={-Latex, thick},
  flow2/.style={Latex-Latex, thick},
  lbl/.style={font=\scriptsize, fill=white, inner sep=1pt}
]
\node[outer, minimum width=10.4cm, minimum height=5.8cm] (elec) at (0,0) {};
\node[font=\bfseries] at ($(elec.north)+(0,-0.35)$) {Boîtier électronique (nuque)};
\node[inner] (xiao) at (0.0,1.15) {XIAO nRF52840};
\node[inner] (l86) at (-3.1,-1.05) {GNSS L86};
\node[inner, minimum width=4.1cm] (bno) at (2.8,-1.05) {BNO085\\(IMU 9 axes + magnétomètre + fusion)};

\node[outer] (smart) at (9.3,0.85) {Smartphone\\(App ENEHOS)};
\node[outer] (ant) at (-4.3,4.3) {Boîtier antenne\\(sommet support)};
\node[outer] (bat) at (-6.3,-5.3) {Boîtier piles AAA};
\node[outer, minimum width=3.0cm] (drv) at (6.2,-5.3) {Driver\\vibreurs};
\node[outer] (vibG) at (4.3,-8.0) {Vibreur gauche};
\node[outer] (vibD) at (8.1,-8.0) {Vibreur droit};

% BLE bidirectionnel (zone dédiée, sans croisement)
\draw[flow2] ($(elec.east)+(0,0.25)$) -- node[lbl, above]{BLE bidirectionnel} ($(smart.west)+(0,0.25)$);
\node[lbl] at ($(elec.east)!0.5!(smart.west)+(0,0.62)$) {Navette MESURES};
\node[lbl] at ($(elec.east)!0.5!(smart.west)+(0,-0.12)$) {Navette VIBRATIONS};

% Liaisons internes
\draw[flow2] (l86.north east) -- node[lbl, above, sloped]{UART TX/RX} (xiao.south west);
\draw[flow] (l86.west) -- ++(-0.9,0) |- node[lbl, left, pos=0.25]{PPS (1 Hz)} (xiao.west);
\draw[flow2] (bno.north west) -- node[lbl, above, sloped]{I\textsuperscript{2}C (+ INT/RST)} (xiao.south east);

% Liaisons externes
\draw[flow] (ant.south) -- node[lbl, left]{RF coax} (l86.north);
\draw[flow] (bat.east) -- node[lbl, above]{Alimentation} ($(elec.south)+(-2.6,0)$);
\draw[flow] ($(elec.south east)+(0.1,0.2)$) -- node[lbl, above]{GPIO/PWM} (drv.north west);
\draw[flow] (drv.south west) -- node[lbl, above, sloped]{canal G} (vibG.north);
\draw[flow] (drv.south east) -- node[lbl, above, sloped]{canal D} (vibD.north);
\end{tikzpicture}
\caption{Architecture cible ENEHOS : un seul boîtier électronique, un lien BLE bidirectionnel, deux navettes applicatives.}
\label{fig:arch}
\end{figure}

\subsection{Fichier navette : structure, horodatage, périodicité}
\begin{boxnote}
Le point logiciel principal à développer est la gestion du \textbf{fichier navette} :
format de payload, stratégie d'horodatage, périodicité d'envoi et robustesse.
\end{boxnote}

\paragraph{\textbf{Contenu minimal des navettes}}
\begin{table}[H]
\centering
\small
\renewcommand{\arraystretch}{1.2}
\begin{tabularx}{\linewidth}{>{\raggedright\arraybackslash}p{3.8cm}
                                >{\raggedright\arraybackslash}p{3.2cm}
                                X}
\toprule
\textbf{Navette} & \textbf{Sens BLE} & \textbf{Contenu} \\
\midrule
Navette MESURES & boîtier \(\rightarrow\) smartphone & Données GNSS (fix, latitude, longitude, vitesse, qualité) + orientation issue du BNO085 + statut système (batterie, erreurs, version). \\
Navette VIBRATIONS & smartphone \(\rightarrow\) boîtier & Commande gauche/droite, activation oui/non, type/pattern de vibration, durée/intensité éventuelle, identifiant de commande. \\
\bottomrule
\end{tabularx}
\end{table}

\subsubsection{Option A : horodater chaque donnée}
Chaque champ mesure embarque son propre timestamp. Précis et flexible, mais payload plus lourd et parsing plus complexe.

\subsubsection{Option B : horodater le fichier (recommandé pour simplicité)}
Une entête navette porte un timestamp global, puis les mesures sont interprétées avec un pas d'échantillonnage connu.
Cette option est plus simple à implémenter côté embarqué et côté application.

\subsubsection{Sens des navettes sur BLE}
\begin{itemize}
  \item \textbf{Navette MESURES} : envoyée par le boîtier via characteristic \texttt{Notify}.
  \item \textbf{Navette VIBRATIONS} : envoyée par le smartphone via characteristic \texttt{Write}.
  \item Le BLE est le \textbf{transport} ; les navettes sont le \textbf{payload applicatif}.
  \item Périodicité à fixer selon usage (exemples : 10 Hz ou 20 Hz pour orientation, 1 Hz pour états lents).
\end{itemize}

\subsection{Photos (à insérer)}
% Mets tes photos dans Overleaf dans un dossier "images/" puis renomme-les proprement.
\begin{figure}[H]
\centering
\begin{subfigure}[t]{0.48\linewidth}
  \centering
  \includegraphics[width=\linewidth]{\imgpath/PHOTO_XIAO_TOP.jpg}
  \caption{XIAO nRF52840 (vue de haut)}
\end{subfigure}\hfill
\begin{subfigure}[t]{0.48\linewidth}
  \centering
  \includegraphics[width=\linewidth]{\imgpath/PHOTO_L86_TOP.jpg}
  \caption{GNSS L86 (vue de haut)}
\end{subfigure}

\vspace{3mm}

\begin{subfigure}[t]{0.48\linewidth}
  \centering
  \includegraphics[width=\linewidth]{\imgpath/PHOTO_BNO085_TOP.jpg}
  \caption{BNO085 (IMU 9 axes + magnétomètre + fusion) (vue de haut)}
\end{subfigure}\hfill
\begin{subfigure}[t]{0.48\linewidth}
  \centering
  \includegraphics[width=\linewidth]{\imgpath/PHOTO_LIAISONS_TOP.jpg}
  \caption{Liaisons (vue de haut)}
\end{subfigure}
\caption{Photos du matériel (remplacer les noms de fichiers par les tiens).}
\label{fig:photos}
\end{figure}

% =========================================================
\section{Matériel de test minimal (banc)}
\begin{boxcheck}
\begin{itemize}
  \item \textbf{Alimentation stabilisée} (mode tension + limitation courant)
  \item \textbf{Oscilloscope} (2 voies minimum)
  \item \textbf{PC} + \textbf{VS Code} (PlatformIO + Python)
  \item Fils Dupont, breadboard, résistances (1k / 4.7k / 10k), capas (100nF / 10µF)
  \item adaptateur \textbf{USB--UART 3.3V} et/ou analyseur logique
\end{itemize}
\end{boxcheck}

% =========================================================
\section{Setup logiciel (VS Code) — base commune}
\begin{boxtest}
\textbf{Objectif :} piloter les tests sans Arduino IDE.
\begin{itemize}
  \item \textbf{PlatformIO} : compilation/flash du XIAO
  \item \textbf{Python} : orchestration, logs, automatisation, rapports
\end{itemize}
\end{boxtest}

\subsection{Structure de repo recommandée}
\begin{verbatim}
enehos-tests/
  firmware/        (PlatformIO: test_harness)
  tests/           (pytest: UART / I2C / BLE)
  logs/            (CSV, JSON)
  images/          (photos matériel, captures oscillo)
  docs/            (Overleaf: ce document)
\end{verbatim}

% =========================================================
\section{Principe HAHIL (Human--AI--Hardware-in-the-Loop)}
% --- Schéma HAHIL (version propre) ---
\begin{figure}[H]
\centering
\begin{tikzpicture}[
  font=\small,
  node distance=18mm and 22mm,
  box/.style={
    draw=black!55,
    rounded corners=2mm,
    very thick,
    fill=black!2, % quasi blanc
    minimum width=45mm,
    minimum height=14mm,
    align=center
  },
  arrow/.style={-Latex, thick, draw=black!70},
  lbl/.style={font=\footnotesize, text=black!80}
]

% Nœuds (ligne principale)
\node[box] (human) {Humain\\\footnotesize(actions physiques)};
\node[box, right=of human] (hw) {Hardware\\\footnotesize(XIAO + capteurs)};
\node[box, right=of hw] (sw) {Software\\\footnotesize(VS Code + scripts)};

% IA sous Software (aligné)
\node[box, below=of sw] (ai) {IA\\\footnotesize(diagnostic + score)};

% Flèches principales
\draw[arrow] (human) -- node[lbl, above]{bouger / brancher / éloigner} (hw);
\draw[arrow] (hw) -- node[lbl, above]{mesures \& logs} (sw);

% Descente vers IA
\draw[arrow] (sw) -- node[lbl, right]{résumé / anomalies} (ai);

% Retour IA -> Humain (propre, en bas)
\draw[arrow] (ai.west) .. controls +(left:30mm) and +(down:12mm) .. 
  node[lbl, below]{recommandations d’actions} (human.south);

\end{tikzpicture}
\caption{Boucle HAHIL : l’humain agit, le hardware mesure, le software collecte, l’IA diagnostique et recommande.}
\label{fig:hahil}
\end{figure}

\begin{boxnote}
\textbf{Idée clé :} l’IA ne remplace pas les mesures. Elle \textbf{classe} et \textbf{explique} rapidement : \emph{câblage}, \emph{alimentation}, \emph{bus bloqué}, \emph{débit instable}, etc.
\end{boxnote}

% =========================================================
% =========================================================
\section{Tests}
\label{sec:tests}

\subsection{Tests par composant}
\label{sec:tests-par-composant}

\begin{boxnote}
Cette sous-section décrit une stratégie de test \textbf{composant par composant} : chaque module est validé \textbf{isolément} avant toute intégration.
Le principe est simple :
\begin{itemize}
  \item un seul élément actif à la fois (pour éviter les faux diagnostics),
  \item des \textbf{preuves mesurables} (oscilloscope, logs série/BLE, mesures d’alimentation),
  \item un résultat clair : \textbf{PASS/FAIL}, accompagné des fichiers de preuve (captures + logs).
\end{itemize}
\end{boxnote}

\begin{boxnote}
Pour chaque composant (XIAO, GNSS L86, BNO085 IMU+magneto+fusion, antenne), on conservera systématiquement :
\begin{itemize}
  \item la configuration matérielle utilisée (câblage + tension d’alim),
  \item la configuration logicielle (firmware / scripts VS Code),
  \item les mesures et calculs réalisés,
  \item les critères d’acceptation (seuils) et les anomalies observées.
\end{itemize}
\end{boxnote}

\subsubsection{\Large XIAO nRF52840}
\label{sec:test-xiao}

\begin{tcolorbox}[colback=white,colframe=black!25,boxrule=0.6pt]
\textbf{But (validation carte seule).} \quad On vérifie :
\begin{itemize}
  \item USB (console série) ;
  \item une sortie GPIO (niveau logique + fréquence mesurée au scope) ;
  \item (option) alimentation externe 3.3 V + limite de courant.
\end{itemize}
\end{tcolorbox}

\vspace{2mm}

\paragraph{\textbf{Matériel minimal}}
\begin{itemize}
  \item XIAO nRF52840 + câble USB-C
  \item Oscilloscope (CH1) + sonde
  \item (Option) alimentation stabilisée 3.3 V
  \item PC + VS Code + PlatformIO
\end{itemize}

\paragraph{\textbf{VS Code / PlatformIO}}
\begin{itemize}
  \item Extensions : \textbf{PlatformIO} + \textbf{Python}
\end{itemize}

\begin{lstlisting}[style=enehos]
pio project init --board seeed_xiao_nrf52840
pio run -t upload
\end{lstlisting}

\paragraph{\textbf{Schéma A — Test GPIO au scope (centré, flèche OK)}}
\begin{figure}[H]
\centering
\begin{tikzpicture}[
  enebox/.style={draw=black!30, rounded corners=3mm, minimum width=55mm, minimum height=14mm, align=center, thick},
  enearrow/.style={-Latex, thick}
]
\node[enebox] (xiao) {\textbf{XIAO nRF52840}};
\node[enebox, right=28mm of xiao] (scope) {\textbf{Oscilloscope}\\CH1};

\draw[enearrow] (xiao.east) -- node[above, font=\footnotesize]{D10} (scope.west);
\draw[enearrow] (xiao.south) .. controls +(down:7mm) and +(down:7mm) ..
node[below, font=\footnotesize]{GND commun} (scope.south);
\end{tikzpicture}
\caption{Mesure GPIO : D10 observée au scope (niveau et fréquence).}
\end{figure}

\paragraph{\textbf{Schéma B — USB (logs) + Option alim externe}}
\begin{figure}[H]
\centering
\begin{tikzpicture}[
  enebox/.style={draw=black!30, rounded corners=3mm, minimum width=55mm, minimum height=14mm, align=center, thick},
  enearrow/.style={-Latex, thick},
  small/.style={font=\footnotesize}
]
\node[enebox] (pc) {\textbf{PC}\\VS Code};
\node[enebox, right=28mm of pc] (xiao) {\textbf{XIAO}\\USB};

\draw[enearrow] (pc.east) -- node[above, small]{USB} (xiao.west);

\node[enebox, below=18mm of xiao] (psu) {\textbf{Alim 3.3 V}\\(option)};
\draw[enearrow] (psu.north) -- node[right, small]{3V3 + GND} (xiao.south);
\end{tikzpicture}
\caption{USB : logs sur PC. Option : alim externe 3.3 V (limite de courant).}
\end{figure}

\paragraph{\textbf{Réglages oscilloscope (GPIO)}}
\begin{tabularx}{\linewidth}{@{}lX@{}}
\toprule
Couplage & DC \\
Vertical & 1 V/div \\
Horizontal & 200 \textmu s/div \\
Trigger & front montant sur CH1 \\
\bottomrule
\end{tabularx}

\paragraph{\textbf{Firmware test (1 kHz + log USB)}}
\begin{lstlisting}[style=enehos,language=C++]
#include <Arduino.h>

void setup() {
  pinMode(D10, OUTPUT);
  Serial.begin(115200);
}

void loop() {
  digitalWrite(D10, HIGH);
  delayMicroseconds(500);
  digitalWrite(D10, LOW);
  delayMicroseconds(500);

  static uint32_t k = 0;
  if ((k++ % 1000) == 0) Serial.println("XIAO: OK");
}
\end{lstlisting}

\paragraph{\textbf{Attendu (mesure + calcul)}}
\[
T = 500\,\mu s + 500\,\mu s = 1\,ms
\qquad \Rightarrow \qquad
f = \frac{1}{T} = 1\,kHz
\]
\begin{itemize}
  \item amplitude : \(\approx 0 \rightarrow 3.3\,V\)
  \item fréquence : \(\approx 1\,kHz\) (tolérance à fixer)
  \item log USB : ``XIAO: OK'' régulier
\end{itemize}

\subsubsection{\Large GNSS Quectel L86}
\label{sec:test-gnss}

\begin{tcolorbox}[colback=white,colframe=black!25,boxrule=0.6pt]
\textbf{But (validation GNSS).} \quad On vérifie :
\begin{itemize}
  \item alimentation + consommation ;
  \item UART : trames NMEA (9600 8N1) ;
  \item PPS : impulsion ~1 Hz au scope ;
  \item FIX : satellites / HDOP / temps d’acquisition.
\end{itemize}
\end{tcolorbox}

\vspace{2mm}

\paragraph{\textbf{Matériel minimal}}
\begin{itemize}
  \item L86 + antenne (selon ton montage)
  \item XIAO nRF52840 (ou USB--UART 3.3 V)
  \item Alimentation stabilisée
  \item Oscilloscope (CH1/CH2)
  \item PC + VS Code
\end{itemize}

\paragraph{\textbf{Schéma A — UART L86 \(\leftrightarrow\) XIAO \(\rightarrow\) PC}}
\begin{figure}[H]
\centering
\begin{tikzpicture}[
  enebox/.style={draw=black!30, rounded corners=3mm, minimum width=55mm, minimum height=14mm, align=center, thick},
  enearrow/.style={-Latex, thick},
  enearrow2/.style={Latex-Latex, thick},
  small/.style={font=\footnotesize}
]
\node[enebox] (l86) {\textbf{GNSS L86}};
\node[enebox, right=5mm of l86] (xiao) {\textbf{XIAO}\\UART$\rightarrow$USB};
\node[enebox, right=5mm of xiao] (pc) {\textbf{PC}\\VS Code};

\draw[enearrow2] (l86.east) -- node[above, small]{TX / RX} (xiao.west);
\draw[enearrow] (xiao.east) -- node[above, small]{USB} (pc.west);
\end{tikzpicture}
\caption{Chaîne de test NMEA : UART du GNSS vers PC.}
\end{figure}

\paragraph{\textbf{Schéma B — PPS vers oscilloscope (lisible)}}
\begin{figure}[H]
\centering
\begin{tikzpicture}[
  enebox/.style={draw=black!30, rounded corners=3mm, minimum width=55mm, minimum height=14mm, align=center, thick},
  enearrow/.style={-Latex, thick}
]
\node[enebox] (l86) {\textbf{GNSS L86}};
\node[enebox, right=28mm of l86] (scope) {\textbf{Oscilloscope}\\CH1};

\draw[enearrow] (l86.east) -- node[above, font=\footnotesize]{PPS} (scope.west);
\draw[enearrow] (l86.south) .. controls +(down:7mm) and +(down:7mm) ..
node[below, font=\footnotesize]{GND commun} (scope.south);
\end{tikzpicture}
\caption{Mesure PPS : la flèche va du GNSS vers le scope (signal).}
\end{figure}

\paragraph{\textbf{Schéma C — Alimentation stabilisée}}
\begin{figure}[H]
\centering
\begin{tikzpicture}[
  enebox/.style={draw=black!30, rounded corners=3mm, minimum width=55mm, minimum height=14mm, align=center, thick},
  enearrow/.style={-Latex, thick},
  small/.style={font=\footnotesize}
]
\node[enebox] (psu) {\textbf{Alim} 3.3 V};
\node[enebox, right=26mm of psu] (l86) {\textbf{GNSS L86}};

\draw[enearrow] (psu.east) -- node[above, small]{VCC} (l86.west);
\draw[enearrow] (psu.south) .. controls +(down:7mm) and +(down:7mm) ..
node[below, small]{GND} (l86.south);
\end{tikzpicture}
\caption{Alimentation : tension fixée + limite courant (sécurité).}
\end{figure}

\paragraph{\textbf{Réglages (alim + oscillo)}}
\begin{tabularx}{\linewidth}{@{}lX@{}}
\toprule
Alimentation & 3.30 V ; limite courant 300 mA (à ajuster ensuite) \\
Oscillo UART (TX) & 1 V/div ; 100 \textmu s/div ; trigger sur f descendant(start bit) \\
Oscillo PPS & 2 V/div ; 200 ms/div ; trigger front montant \\
UART & 9600 bps ; 8N1 \\
\bottomrule
\end{tabularx}

\paragraph{\textbf{Firmware XIAO : passerelle UART \(\rightarrow\) USB}}
\begin{lstlisting}[style=enehos,language=C++]
#include <Arduino.h>

void setup() {
  Serial.begin(115200);  // USB
  Serial1.begin(9600);   // GNSS (defaut)
}

void loop() {
  while (Serial1.available()) {
    Serial.write(Serial1.read());
  }
}
\end{lstlisting}

\paragraph{\textbf{Attendu (mesure + calcul)}}
PPS :
\[
T_{\mathrm{PPS}} \approx 1\,s
\qquad \Rightarrow \qquad
f_{\mathrm{PPS}} \approx 1\,Hz
\]
\begin{itemize}
  \item UART : trames NMEA visibles en continu (\$GxGGA, \$GxRMC, …)
  \item PPS : impulsion stable (1 Hz)
  \item FIX : satellites détectés (seuils à définir ensuite)
\end{itemize}

\subsubsection{\Large BNO085 (IMU 9 axes + magnétomètre + fusion)}
\label{sec:test-imu}

\begin{tcolorbox}[colback=white,colframe=black!25,boxrule=0.6pt]
\textbf{But (validation BNO085 IMU+magneto+fusion).} \quad On vérifie :
\begin{itemize}
  \item alimentation + stabilité
  \item bus I\textsuperscript{2}C : détection (ACK) + échanges stables
  \item génération de données : accel/gyro/quaternion (cohérence simple)
  \item (option) ligne INT : activité / fréquence d’interrupt
\end{itemize}
\end{tcolorbox}

\vspace{2mm}

\paragraph{\textbf{Matériel minimal}}
\begin{itemize}
  \item BNO085 (IMU 9 axes + magnétomètre + fusion) (breakout)
  \item XIAO nRF52840
  \item Alimentation (USB XIAO suffit souvent) + (option) alim externe 3.3 V
  \item Oscilloscope (CH1/CH2)
  \item PC + VS Code + PlatformIO
\end{itemize}

\paragraph{\textbf{Schéma A — I\textsuperscript{2}C BNO085 \(\rightarrow\) XIAO \(\rightarrow\) PC}}
\begin{figure}[H]
\centering
\begin{tikzpicture}[
  enebox/.style={draw=black!30, rounded corners=3mm, minimum width=55mm, minimum height=14mm, align=center, thick},
  enearrow/.style={-Latex, thick},
  enearrow2/.style={Latex-Latex, thick},
  small/.style={font=\footnotesize}
]
\node[enebox] (imu) {\textbf{BNO085}\\\textbf{IMU+magneto+fusion}};
\node[enebox, right=10mm of imu] (xiao) {\textbf{XIAO}\\I\textsuperscript{2}C$\rightarrow$USB};
\node[enebox, right=6mm of xiao] (pc) {\textbf{PC}\\VS Code};

\draw[enearrow2] (imu.east) -- node[above, small]{SDA / SCL} (xiao.west);
\draw[enearrow] (xiao.east) -- node[above, small]{USB} (pc.west);
\end{tikzpicture}
\caption{Chaîne de test BNO085 : I\textsuperscript{2}C vers XIAO, logs sur PC.}
\end{figure}

\paragraph{\textbf{Schéma B — Mesure I\textsuperscript{2}C au scope (SCL)}}
\begin{figure}[H]
\centering
\begin{tikzpicture}[
  enebox/.style={draw=black!30, rounded corners=3mm, minimum width=55mm, minimum height=14mm, align=center, thick},
  enearrow/.style={-Latex, thick},
  small/.style={font=\footnotesize}
]
\node[enebox] (bus) {\textbf{Bus I\textsuperscript{2}C}\\(SCL/SDA)};
\node[enebox, right=28mm of bus] (scope) {\textbf{Oscilloscope}\\CH1};

\draw[enearrow] (bus.east) -- node[above, small]{SCL (ou SDA)} (scope.west);
\end{tikzpicture}
\caption{Mesure I\textsuperscript{2}C : vérifier niveaux et activité sur SCL/SDA.}
\end{figure}

\paragraph{\textbf{Réglages (alim + oscillo)}}
\begin{tabularx}{\linewidth}{@{}lX@{}}
\toprule
Alimentation & 3.3 V (ou via 3V3 XIAO) ; limite 200 mA (si alim externe) \\
Oscillo I\textsuperscript{2}C & 1 V/div ; 10 \textmu s/div (à ajuster) ; trigger front montant \\
I\textsuperscript{2}C (départ) & 100 kHz (mode debug) \\
\bottomrule
\end{tabularx}

\paragraph{\textbf{Firmware XIAO : scan I\textsuperscript{2}C + stream brut}}
\begin{lstlisting}[style=enehos,language=C++]
#include <Arduino.h>
#include <Wire.h>

void setup() {
  Serial.begin(115200);
  Wire.begin(); // SDA/SCL selon mapping XIAO
  Wire.setClock(100000);
  delay(200);

  // Scan I2C simple
  Serial.println("I2C scan:");
  for (uint8_t a = 1; a < 127; a++) {
    Wire.beginTransmission(a);
    if (Wire.endTransmission() == 0) {
      Serial.print("  found 0x");
      Serial.println(a, HEX);
    }
  }
}

void loop() {
  // placeholder : on remplacera par lecture BNO085 (SH-2 / lib)
  delay(1000);
}
\end{lstlisting}

\paragraph{\textbf{Attendu (mesure + calcul)}}
\begin{itemize}
  \item I\textsuperscript{2}C : au moins une adresse répond (ACK) (adresse exacte à confirmer selon configuration)
  \item SCL/SDA : repos à \(\approx 3.3\,V\), transitions nettes, activité pendant les lectures
  \item (suite) streaming BNO085 : fréquence d’échantillonnage définie + log stable (sera détaillé ensuite)
\end{itemize}

% =========================================================
\subsubsection{\Large Antenne GNSS active}
\label{sec:test-antenne}

\begin{tcolorbox}[colback=white,colframe=black!25,boxrule=0.6pt]
\textbf{But (comparatif antenne).} \quad On compare le GNSS :
\begin{itemize}
  \item sans antenne active / config A
  \item avec antenne GNSS active / config B
\end{itemize}
L’objectif est d’obtenir une amélioration mesurable (satellites, HDOP, TTFF).
\end{tcolorbox}

\vspace{2mm}

\paragraph{\textbf{Matériel minimal}}
\begin{itemize}
  \item GNSS L86 (même montage que le test GNSS)
  \item Antenne GNSS active + câble/connexion RF du montage
  \item XIAO (ou USB--UART) + PC
  \item (Option) oscilloscope pour vérifier PPS identique en A/B
\end{itemize}

\paragraph{\textbf{Schéma A — A/B testing (même chaîne, antenne changée)}}
\begin{figure}[H]
\centering
\begin{tikzpicture}[
  enebox/.style={draw=black!30, rounded corners=3mm, minimum width=50mm, minimum height=14mm, align=center, thick},
  enearrow/.style={-Latex, thick},
  small/.style={font=\footnotesize}
]
\node[enebox] (ant) {\textbf{Antenne}\\(A ou B)};
\node[enebox, right=5mm of ant] (l86) {\textbf{GNSS L86}};
\node[enebox, right=18mm of l86] (pc) {\textbf{PC}\\Logs};

\draw[enearrow] (ant.east) -- node[above, small]{RF} (l86.west);
\draw[enearrow] (l86.east) -- node[above, small]{UART/USB} (pc.west);
\end{tikzpicture}
\caption{Comparatif : seule l’antenne change, tout le reste est identique.}
\end{figure}

\paragraph{\textbf{Table de mesures A/B (à remplir)}}
\begin{tabularx}{\linewidth}{@{}lXXXX@{}}
\toprule
\textbf{Config} & \textbf{TTFF (s)} & \textbf{Nb satellites} & \textbf{HDOP} & \textbf{Stabilité (dropouts)} \\
\midrule
A (référence) &  &  &  &  \\
B (antenne active) &  &  &  &  \\
\bottomrule
\end{tabularx}

\paragraph{\textbf{Protocole comparatif}}
\begin{itemize}
  \item Conditions identiques : lieu, orientation, durée, alimentation
  \item Lancer un log GNSS pendant une fenêtre de temps fixe (ex : 5 min)
  \item Relever TTFF, satellites, HDOP, pertes de FIX
\end{itemize}

\paragraph{\textbf{Attendu (calculs)}}
Amélioration relative (exemple pour TTFF) :
\[
\Delta_{\%} = 100 \times \frac{TTFF_A - TTFF_B}{TTFF_A}
\]
Même logique pour satellites (+) et HDOP (-).

\subsection{Tests par liaison}
\label{sec:tests-par-liaison}

\begin{tcolorbox}[colback=white,colframe=black!25,boxrule=0.6pt]
\textbf{Principe.} \quad Ici on ne teste plus un composant, on teste \textbf{une liaison} (un bus / un lien) :
\begin{itemize}
  \item \textbf{électrique} (niveaux 0/3.3 V, pull-ups, bruit, fronts),
  \item \textbf{protocole} (baudrate UART, ACK I\textsuperscript{2}C, advertising BLE),
  \item \textbf{performance} (débit, latence, pertes),
  \item \textbf{robustesse} (débranche/rebranche, reset, stress 5 min).
\end{itemize}
Chaque test de liaison produit un verdict \textbf{PASS/FAIL} + des preuves (captures oscillo + logs).
\end{tcolorbox}

\vspace{2mm}

% =========================================================
\paragraph{\textbf{Résumé global des liaisons (nature + sens)}}

\begin{figure}[H]
\centering
\begin{tikzpicture}[
  x=1cm,y=1cm,
  scale=0.90,
  transform shape,
  enebox/.style={draw=black!30, rounded corners=3mm, minimum width=3.4cm, minimum height=1.1cm, align=center, thick},
  innerbox/.style={draw=black!35, rounded corners=2mm, minimum width=2.9cm, minimum height=0.9cm, align=center, fill=black!5},
  enearrow/.style={-Latex, thick},
  enearrow2/.style={Latex-Latex, thick},
  small/.style={font=\scriptsize, fill=white, inner sep=1pt}
]
\node[enebox, minimum width=10.4cm, minimum height=5.8cm] (elec) at (0,0) {};
\node[font=\footnotesize\bfseries] at ($(elec.north)+(0,-0.35)$) {Boîtier électronique unique (nuque)};
\node[innerbox] (xiao) at (0.0,1.15) {\textbf{XIAO nRF52840}};
\node[innerbox] (l86) at (-3.1,-1.05) {\textbf{GNSS L86}};
\node[innerbox, minimum width=4.0cm] (bno) at (2.8,-1.05) {\textbf{BNO085}\\(IMU+magneto+fusion)};

\node[enebox] (phone) at (9.3,0.85) {\textbf{Smartphone}\\App ENEHOS};
\node[enebox] (ant) at (-4.3,4.3) {\textbf{Boîtier antenne}};
\node[enebox] (bat) at (-6.3,-5.3) {\textbf{Boîtier piles AAA}};
\node[enebox, minimum width=3.0cm] (drv) at (6.2,-5.3) {\textbf{Driver}\\vibreurs};
\node[enebox] (vibG) at (4.3,-8.0) {\textbf{Vibreur gauche}};
\node[enebox] (vibD) at (8.1,-8.0) {\textbf{Vibreur droit}};

% BLE unique, deux flux navette
\draw[enearrow2] ($(elec.east)+(0,0.25)$) -- node[small, above]{BLE bidirectionnel} ($(phone.west)+(0,0.25)$);
\node[small] at ($(elec.east)!0.5!(phone.west)+(0,0.62)$) {Navette MESURES};
\node[small] at ($(elec.east)!0.5!(phone.west)+(0,-0.12)$) {Navette VIBRATIONS};

% Liaisons internes boîtier électronique
\draw[enearrow2] (l86.north east) -- node[small, above, sloped]{UART TX/RX} (xiao.south west);
\draw[enearrow] (l86.west) -- ++(-0.9,0) |- node[small, left, pos=0.25]{PPS (1 Hz)} (xiao.west);
\draw[enearrow2] (bno.north west) -- node[small, above, sloped]{I\textsuperscript{2}C (+ INT/RST)} (xiao.south east);

% Liaisons périphériques
\draw[enearrow] (ant.south) -- node[small, left]{RF coax} (l86.north);
\draw[enearrow] (bat.east) -- node[small, above]{Alimentation} ($(elec.south)+(-2.6,0)$);
\draw[enearrow] ($(elec.south east)+(0.1,0.2)$) -- node[small, above]{GPIO/PWM} (drv.north west);
\draw[enearrow] (drv.south west) -- node[small, above, sloped]{canal G} (vibG.north);
\draw[enearrow] (drv.south east) -- node[small, above, sloped]{canal D} (vibD.north);

\end{tikzpicture}
\caption{Vue synthétique cible : boîtier électronique unique avec L86+BNO085 internes, navettes BLE bidirectionnelles, UART TX/RX, I\textsuperscript{2}C et PPS.}
\end{figure}

\vspace{2mm}

% =========================================================
\paragraph{\textbf{Table récapitulative des liaisons}}

\begin{table}[H]
\centering
\small
\renewcommand{\arraystretch}{1.2}
\setlength{\tabcolsep}{4pt}

\begin{tabularx}{\linewidth}{>{\raggedright\arraybackslash}p{2.2cm}
                                >{\raggedright\arraybackslash}p{2.5cm}
                                >{\raggedright\arraybackslash}p{2cm}
                                X}
\toprule
\textbf{Liaison} & \textbf{Nature} & \textbf{Niveau} & \textbf{Rôle} \\
\midrule
UART & filaire numérique série & TTL 3.3 V & GNSS L86 \(\leftrightarrow\) XIAO (TX/RX, trames NMEA) \\
PPS & impulsion numérique & 3.3 V & Synchronisation temporelle 1 Hz \\
I\textsuperscript{2}C & bus 2 fils (SDA/SCL) & 3.3 V + pull-ups & BNO085 (IMU+magneto+fusion) \(\leftrightarrow\) XIAO \\
BLE (bidirectionnel) & radio 2.4 GHz & N/A & Transport des navettes : mesures \(\leftrightarrow\) commandes \\
RF GNSS & radio L1 + coax & N/A & Boîtier antenne \(\rightarrow\) L86 \\
Commande vibreurs & sorties GPIO/PWM + driver & 3.3 V (pilotage driver) & XIAO \(\rightarrow\) vibreur gauche/droit \\
Alimentation & filaire puissance & selon design (AAA) & Boîtier piles AAA \(\rightarrow\) boîtier électronique \\
\bottomrule
\end{tabularx}

\end{table}
\vspace{4mm}

% =========================================================
\paragraph{\textbf{Rappel des broches utilisées (boîtier électronique unique ENEHOS)}}

\begin{table}[H]
\centering
\small
\renewcommand{\arraystretch}{1.2}
\setlength{\tabcolsep}{4pt}

\begin{tabularx}{\linewidth}{>{\raggedright\arraybackslash}p{2.5cm}
                                >{\raggedright\arraybackslash}p{2cm}
                                >{\raggedright\arraybackslash}p{2cm}
                                X}
\toprule
\textbf{Sous-système (nuque)} & \textbf{Signal} & \textbf{Broche XIAO} & \textbf{Commentaire} \\
\midrule
GNSS L86 & TX → RX & D7 (RX) & UART GNSS vers XIAO \\
GNSS L86 & RX ← TX & D6 (TX) & UART XIAO vers GNSS \\
GNSS L86 & PPS & D2 & Impulsion 1 Hz \\
GNSS L86 & RST & D3 & Reset GNSS \\
GNSS L86 & F-ON & D0 (option) & Activation module \\
BNO085 (IMU+magneto+fusion) & SDA & D4 (SDA) & Bus I\textsuperscript{2}C \\
BNO085 (IMU+magneto+fusion) & SCL & D5 (SCL) & Bus I\textsuperscript{2}C \\
BNO085 (IMU+magneto+fusion) & \texttt{!INT} & D9 & Interruption capteur \\
BNO085 (IMU+magneto+fusion) & \texttt{!RST} & D8 & Reset capteur \\
Driver vibreur gauche & CMD\_L & GPIO/PWM libre (\`a affecter) & Commande moteur gauche \\
Driver vibreur droit & CMD\_R & GPIO/PWM libre (\`a affecter) & Commande moteur droit \\
\bottomrule
\end{tabularx}

\end{table}

\vspace{4mm}

% =========================================================
\paragraph{\textbf{Règle d’or (diagnostic rapide)}}
\begin{itemize}
  \item Si \textbf{UART} ne sort rien : vérifier TX/RX inversés, GND commun, baudrate, niveaux 3.3 V.
  \item Si \textbf{I\textsuperscript{2}C} ne répond pas : vérifier pull-ups, SDA/SCL inversés, bus bloqué (SDA low).
  \item Si \textbf{BLE} est instable : vérifier distance, alimentation, fréquence de notifications trop élevée.
  \item Si \textbf{PPS} absent : pas de FIX ou PPS pas câblé/configuré.
\end{itemize}

\subsubsection{\Large UART — Liaison GNSS L86 ↔ XIAO}
\label{sec:test-uart}

\paragraph{\textbf{1. Nature électrique}}

\begin{itemize}
\item Type : UART asynchrone (8N1)
\item Niveau : CMOS / TTL 3.3 V
\item Pas de RS232 direct (interdit sans level shifter)
\item Baudrate par défaut L86 : 9600 bps :contentReference[oaicite:2]{index=2}
\end{itemize}

\paragraph{\textbf{2. Brochage exact prototype ENEHOS}}

\begin{tabular}{ll}
TX L86 (pin 2) & → D7 (RX XIAO) \\
RX L86 (pin 1) & ← D6 (TX XIAO) \\
GND & ↔ GND commun \\
VCC L86 & 3.3 V stable \\
PPS (option) & → D2 \\
RST (option) & → D3 \\
F-ON (option) & → D0 \\
\end{tabular}

\paragraph{\textbf{3. Stack VS Code}}

\begin{itemize}
\item VS Code
\item PlatformIO
\item Board : seeed_xiao_nrf52840
\item Serial Monitor : 115200 bps
\end{itemize}

\begin{lstlisting}[style=enehos,language=C++]
#include <Arduino.h>

void setup() {
  Serial.begin(115200);   // USB debug
  Serial1.begin(9600);    // GNSS L86
}

void loop() {
  while (Serial1.available()) {
    Serial.write(Serial1.read());
  }
}
\end{lstlisting}

\paragraph{\textbf{4. Réglages oscilloscope (diagnostic électrique)}}

\begin{itemize}
\item Sonde CH1 sur TX L86
\item GND commun
\item 1 V/div
\item 100 µs/div
\item Trigger front descendant (start bit)
\end{itemize}

Fréquence bit attendue :

\[
T_{bit} = \frac{1}{9600} = 104\,\mu s
\]

\paragraph{\textbf{5. Résultats attendus}}

\begin{itemize}
\item Trames NMEA visibles :
\texttt{\$GxGGA}, \texttt{\$GxRMC}, etc.
\item Pas de caractères corrompus
\item Niveaux logiques : 0V / 3.3V
\item Débit stable
\end{itemize}

\paragraph{\textbf{6. Test robustesse}}

\begin{itemize}
\item Débrancher RX → aucune donnée
\item Inverser TX/RX → aucune trame
\item Modifier baudrate → trames illisibles
\end{itemize}

PASS si trames continues 5 min sans erreur.

\subsubsection{\Large I\textsuperscript{2}C — Liaison BNO085 (IMU+magneto+fusion) ↔ XIAO}
\label{sec:test-i2c}

\paragraph{\textbf{1. Nature électrique}}

\begin{itemize}
\item Bus maître-esclave
\item Lignes : SDA / SCL
\item Niveau : 3.3 V
\item Open-drain + résistances pull-up
\end{itemize}

\paragraph{\textbf{2. Brochage prototype ENEHOS}}

\begin{tabular}{ll}
SDA BNO085 & → D4 (SDA) \\
SCL BNO085 & → D5 (SCL) \\
!INT & → D9 \\
!RST & → D8 \\
VIN & → 3V3 \\
GND & → GND \\
\end{tabular}

\paragraph{\textbf{3. Stack VS Code}}

\begin{lstlisting}[style=enehos,language=C++]
#include <Arduino.h>
#include <Wire.h>

void setup() {
  Serial.begin(115200);
  Wire.begin();
  Wire.setClock(100000);
  delay(200);

  Serial.println("Scan I2C");
  for(uint8_t addr=1; addr<127; addr++){
    Wire.beginTransmission(addr);
    if(Wire.endTransmission()==0){
      Serial.print("Found: 0x");
      Serial.println(addr,HEX);
    }
  }
}

void loop(){}
\end{lstlisting}

\paragraph{\textbf{4. Oscilloscope}}

\begin{itemize}
\item CH1 : SCL
\item CH2 : SDA
\item 1 V/div
\item 10 µs/div
\item Trigger sur SCL
\end{itemize}

\paragraph{\textbf{5. Attendu}}

\begin{itemize}
\item SDA et SCL au repos : 3.3 V
\item Pulses propres
\item ACK visible après adresse
\item Adresse BNO085 détectée
\end{itemize}

\paragraph{\textbf{6. Tests erreurs}}

\begin{itemize}
\item SDA bloqué à 0 → bus stuck
\item Pull-ups absentes → fronts lents
\item Clock trop élevé → pertes ACK
\end{itemize}

PASS si scan stable + données orientation continues.


\subsubsection{\Large BLE — Liaison XIAO ↔ Smartphone (navettes)}
\label{sec:test-ble}

\paragraph{\textbf{1. Nature}}

\begin{itemize}
\item Radio 2.4 GHz (Bluetooth Low Energy)
\item \textbf{Un seul lien BLE bidirectionnel}
\item \textbf{Deux flux applicatifs} : navette mesures et navette commande vibrations
\end{itemize}

\paragraph{\textbf{2. Architecture logique ENEHOS}}

\begin{itemize}
\item Service UUID : ENEHOS (à figer)
\item Characteristic Notify : \textbf{navette mesures horodatée} (boîtier \(\rightarrow\) smartphone)
\item Characteristic Write : \textbf{navette commande vibrations} (smartphone \(\rightarrow\) boîtier)
\item Les deux flux existent sur le même BLE (ce ne sont pas deux BLE capteurs séparés)
\end{itemize}

\paragraph{\textbf{3. Stack VS Code}}

\begin{itemize}
\item ArduinoBLE library
\item PlatformIO
\end{itemize}

\begin{lstlisting}[style=enehos,language=C++]
#include <ArduinoBLE.h>

BLEService enehosService("180A");
BLECharacteristic navetteMesuresChar("2A57", BLENotify, 80);
BLECharacteristic navetteCmdChar("2A58", BLEWrite | BLEWriteWithoutResponse, 80);

const char navetteMesures[] =
  "ts=2026-02-16T10:00:00Z;gnss=fix,lat,lon,spd;imu=yaw,pitch,roll;status=ok";

void setup() {
  Serial.begin(115200);
  BLE.begin();
  BLE.setLocalName("ENEHOS_NECK");
  BLE.setAdvertisedService(enehosService);
  enehosService.addCharacteristic(navetteMesuresChar);
  enehosService.addCharacteristic(navetteCmdChar);
  BLE.addService(enehosService);
  BLE.advertise();
}

void loop() {
  BLEDevice central = BLE.central();
  if (central) {
    while (central.connected()) {
      navetteMesuresChar.writeValue(
        (const uint8_t*)navetteMesures, sizeof(navetteMesures) - 1
      );

      if (navetteCmdChar.written()) {
        uint8_t cmdBuf[80];
        int n = navetteCmdChar.valueLength();
        navetteCmdChar.readValue(cmdBuf, n);
        // parser navette commande vibrations : gauche/droite, type, duree
      }

      delay(100); // exemple: 10 Hz
    }
  }
}
\end{lstlisting}

\paragraph{\textbf{4. Tests}}

\begin{itemize}
\item Scan smartphone → device visible
\item Connexion < 3 s
\item Notifications "navette mesures" reçues à la cadence attendue
\item Commandes "navette vibrations" correctement reçues côté boîtier
\item Perte connexion test (distance)
\end{itemize}

\paragraph{\textbf{5. Mesures}}

\begin{itemize}
\item RSSI
\item Latence aller (mesures) / retour (commandes)
\item Taux perte paquets
\item Gigue de périodicité des navettes
\item Stabilité 10 min
\end{itemize}

\paragraph{\textbf{6. Critère PASS}}

\begin{itemize}
\item Connexion stable
\item Navette mesures continue (pas de trou bloquant)
\item Navette commande vibrations exécutée sans ambiguïté
\item Pas de freeze
\item Pas de reboot XIAO
\end{itemize}

% =========================================================
\section{Estimation de charge (pour planning)}

\begin{table}[H]
\centering
\small
\renewcommand{\arraystretch}{1.2}
\setlength{\tabcolsep}{4pt}
\begin{tabularx}{\linewidth}{>{\raggedright\arraybackslash}p{6.2cm}
                                >{\raggedright\arraybackslash}p{2.2cm}
                                X}
\toprule
\textbf{Lot} & \textbf{Charge} & \textbf{Commentaire} \\
\midrule
Navette : format + horodatage & 8 h & Spécification payload, option A/B, versionnement navette \\
BLE : TX mesures + RX commandes & 10 h & Service/characteristics, bufferisation, erreurs de transport \\
Intégration GNSS & 6 h & Trames, qualité fix, cohérence horodatage \\
Intégration BNO085 (IMU+magneto+fusion) & 8 h & Orientation fusionnée, fréquence, vérifications de cohérence \\
Driver vibreurs & 6 h & Mapping commandes, sécurité activation, tests GPIO/PWM \\
Tests terrain + logs + validation & 12 h & Scénarios dynamiques, collecte traces, validation critères \\
Documentation + mise en page & 4 h & Mise à jour du document, schémas, versionnage \\
\bottomrule
\end{tabularx}
\end{table}
% =========================================================
\section{Annexes (captures oscillo, logs, photos)}
% Tu ajouteras ici les captures d’oscillo, screenshots de logs, etc.

\end{document}
